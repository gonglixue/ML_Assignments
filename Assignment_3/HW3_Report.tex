% !TEX TS-program = pdflatex
% !TEX encoding = UTF-8 Unicode

% This is a simple template for a LaTeX document using the "article" class.
% See "book", "report", "letter" for other types of document.

\documentclass[11pt]{article} % use larger type; default would be 10pt

\usepackage[utf8]{inputenc} % set input encoding (not needed with XeLaTeX)

%%% Examples of Article customizations
% These packages are optional, depending whether you want the features they provide.
% See the LaTeX Companion or other references for full information.

%%% PAGE DIMENSIONS
\usepackage{geometry} % to change the page dimensions
\geometry{a4paper} % or letterpaper (US) or a5paper or....
% \geometry{margin=2in} % for example, change the margins to 2 inches all round
% \geometry{landscape} % set up the page for landscape
%   read geometry.pdf for detailed page layout information

\usepackage{graphicx} % support the \includegraphics command and options

% \usepackage[parfill]{parskip} % Activate to begin paragraphs with an empty line rather than an indent

%%% PACKAGES
\usepackage{booktabs} % for much better looking tables
\usepackage{array} % for better arrays (eg matrices) in maths
\usepackage{paralist} % very flexible & customisable lists (eg. enumerate/itemize, etc.)
\usepackage{verbatim} % adds environment for commenting out blocks of text & for better verbatim
\usepackage{subfig} % make it possible to include more than one captioned figure/table in a single float
% These packages are all incorporated in the memoir class to one degree or another...

%%% HEADERS & FOOTERS
\usepackage{fancyhdr} % This should be set AFTER setting up the page geometry
\pagestyle{fancy} % options: empty , plain , fancy
\renewcommand{\headrulewidth}{0pt} % customise the layout...
\lhead{}\chead{}\rhead{}
\lfoot{}\cfoot{\thepage}\rfoot{}

%%% SECTION TITLE APPEARANCE
\usepackage{sectsty}
\allsectionsfont{\sffamily\mdseries\upshape} % (See the fntguide.pdf for font help)
% (This matches ConTeXt defaults)

%%% ToC (table of contents) APPEARANCE
\usepackage[nottoc,notlof,notlot]{tocbibind} % Put the bibliography in the ToC
\usepackage[titles,subfigure]{tocloft} % Alter the style of the Table of Contents
\renewcommand{\cftsecfont}{\rmfamily\mdseries\upshape}
\renewcommand{\cftsecpagefont}{\rmfamily\mdseries\upshape} % No bold!

\usepackage[colorlinks,linkcolor=red]{hyperref}


%%% END Article customizations

%%% The "real" document content comes below...

\title{HW3 Report}
\author{Gong Lixue}
%\date{} % Activate to display a given date or no date (if empty),
         % otherwise the current date is printed 

\begin{document}
\maketitle

\section{Neural Networks}

The implementation of feedforward and backpropagation process is shown in code folder.
At the end of training iterations, $ loss = 1.717e-01 $, $ accuracy = 0.9400 $, which is not the optimal actually.
And for testing, $ loss = 2.522e-01 $, and $accuracy=0.9230$.

\section{K-Neareast Neighbor}
(a) boundary figures is shown below:

\begin{figure}[h]
\centering
\includegraphics[width=2in]{knn_k1.jpg}  %图片名
\caption{K = 1}
\label{fig1}
\end{figure}

\begin{figure}[h]
\centering
\includegraphics[width=2in]{knn_k10.jpg}  %图片名
\caption{K = 10}
\label{fig2}
\end{figure}

\begin{figure}[h]
\centering
\includegraphics[width=2in]{knn_k100.jpg}  %图片名
\caption{K = 100}
\label{fig3}
\end{figure}

(b) 
One of methods is that choosing a proper K by Cross-Validation. Compute the validation error on validation set using different values for K. And pick the optimal K with the lowest validation error.

(c)
Firstly, I wrote a simple python script to fetch and save check code images from this \href{http://jwbinfosys.zju.edu.cn/default2.aspx}{website} automatically.(See Figure 4)

\begin{figure}
\centering
\includegraphics[height=3in]{knn_spider.jpg}  %图片名
\caption{images fetched by python spider}
\label{fig4}
\end{figure}

And then, label these images by hand. I use 100 images of them. The raw labels are recored in file  $./knn/hack_py/label\_100img.txt$. Each row in this file means the actual codes that corresponding image represents.

Before recognizing a check code image, we should generate a $.mat$ file which is used for training in knn. Run $gen\_hack\_data.m$ to generate a .mat file.
Finally, we can test the algorithm to recognize a check code image(see $knn\_exp.m$ Part2 and Figure 5).

\begin{figure}
\centering
\includegraphics[width=2in]{knn_showimg.jpg}  %图片名
\caption{show\_image}
\label{fig5}
\end{figure}

\begin{figure}
\centering
\includegraphics[width=2in]{knn_resultdigits.jpg}  %图片名
\caption{result digits}
\label{fig6}
\end{figure}


\section{Decision Tree and ID3}
The decision tree and respective information gain is illustrated as the graph below:

\begin{figure}[h]
\centering
\includegraphics[width=2in]{tree.png}  %图片名
\caption{decision tree}
\label{fig7}
\end{figure}

\section{K-Means Clustering}

(a) The visualization of process with smallest SD:
\begin{center}
\includegraphics[width=2in]{kmeans_min.jpg}  %图片名

Figure 8: kmeans process with smallest SD
\end{center}



The visualization of process with largest SD:
\begin{center}
\includegraphics[width=2in]{kmeans_max.jpg}  %图片名

Figure 9: kmeans process with largest SD
\end{center}

(b) 
We can run it from multiple starting points, and pick the solution with the smallest SD.

(c)
The visualization of centroid is illustrated below. And we can find that the cluster center can represent the patterns in dataset.

\begin{center}
\includegraphics[width=2in]{kmeans_k10.jpg}  %图片名

Figure 10: k=10
\end{center}

\begin{center}
\includegraphics[width=2in]{kmeans_k20.jpg}  %图片名

Figure 11: k=50
\end{center}

\begin{center}
\includegraphics[width=2in]{kmeans_k50.jpg}  %图片名

Figure 12: k=50
\end{center}


(d)
This is the original image:
\begin{center}
\includegraphics[width=2in]{aa_origin.jpg}  %图片名

Figure 13: original image
\end{center}


And wen can obtain the compressed images after runing $vq.m$:

\begin{center}
\includegraphics[width=2in]{aa_8.jpg}  %图片名

Figure 14: K = 8
\end{center}

\begin{center}
\includegraphics[width=2in]{aa_16.jpg}  %图片名

Figure 15: K = 16
\end{center}

\begin{center}
\includegraphics[width=2in]{aa_32.jpg}  %图片名

Figure 16: K = 32
\end{center}

We can observe that when we set K to 64, these is no obvious change comparing to original image. When K=64, each pixel can be represented with $log(K) = 8$ bits. So the data is not efficiently compressed actually.


\end{document}




